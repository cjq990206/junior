\section{四、系统非功能性需求分析}
\subsection{4.1 可靠性}
影视舆情分析、预警与监测系统要求能在 24 小时内一直稳定运行。当发生某些热点事件时,随着系统使用人数的不断增长,客户端向服务端发送的请求也越来越多,如果只有一台应用服务器来接受来自很多客户端的请求,一旦资源耗尽,发生宕机的可能性很大,所以系统的可靠性十分重要。
\subsubsection{4.1.1 易恢复性}
本系统发生故障后,系统应重建其性能水平并恢复直接受影响数据的能力。发布新版本时,要做好回滚方案,以备异常紧急处理。同时做好备份,系统监控的字段以及历史查询信息误删除时可进行恢复。
\subsubsection{4.1.2 容错性}
在系统出错时,不影响用户的行为操作与数据。在设计数据库的时候,可以进行冗余设计,采用主从数据库的方式,把读操作和写操作分离,部署在不同的服务器上,从数据库主要用来查询数据用,不进行写入操作,主数据库为写库,用来写入和更新数据,每次当主数据库有写的操作时,数据同步到从数据库去。设计多个从数据库,即拥有了多个容灾副版本,当主数据库服务器宕机的时候,马上切换到其中一台正常运行的从数据库服务器去,提高了整个系统的容错性。
\subsubsection{4.1.3 成熟性}
系统故障率需要保持在一定水平以下。在设计系统的时候,可以考虑部署一台负载均衡服务器,把请求合理的分配到多台应用服务器上去,达到资源的合理分配。Nginx 是一个可以用来做反向代理的负载均衡的轻量级软件,占用内存小,带宽低,并发连接数大,配置简单,功能强大,还可以缓存静态资源,例如图片等。具体设计时,可以部署一台 Nginx反向代理服务器作为接受客户端请求的统一访问接口点,根据业务量的多少来设置集群服务器的个数,起到负载均衡的效果。
\subsection{4.2 易用性}
易用性是以用户为中心,结合视觉、交互、情感等综合感受,使产品符合用户习惯的能力以及其对使用的期望。 它会对用户使用产品的生产效率、错误率以及用户对新产品的接收程度产生很大的影响。
\subsubsection{4.2.1 易学习性}
系统学习成本低。本系统无需学习即可使用。
\subsubsection{4.2.2 易操作性}
系统建设过程中,系统涵盖了完整的业务需求。模块在组件各部分之间实现信息的顺畅流动,系统具有连贯性,交互设置合理,功能明确清晰。
\subsubsection{4.2.3 用户错误防御机制}
我们设计了上百种错误防御机制,系统遇到错误的输入时会触发检测机制,提示用户输入错误,防止造成系统崩溃影响用户使用体验。
\subsubsection{4.2.4 用户界面美观}
本系统采用大屏数据展示界面,进行了web界面原型设计并经用户反复确认,并且通过了性能测试、系统测试及用户接受测试。
