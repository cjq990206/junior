\section{系统功能性需求分析}

\begin{figure}[!htb]
	\centering
	\includegraphics[scale=1]{image/logo1.png}
	\caption{北京交通大学校徽}
\end{figure}



\begin{longtable}[c]{c|ccc}
	\caption{用户登录用例表}
	\label{tab:my-table}\\
	\shline
	\multicolumn{1}{c|}{\textbf{用例编号}} & \multicolumn{1}{c|}{UC-01} & \multicolumn{1}{c|}{用例名称} &  用户登录\\ \hline
	\endhead
	%
	\multicolumn{1}{c|}{\textbf{活动者}} & \multicolumn{1}{c|}{用户} & \multicolumn{1}{c|}{优先级} &高  \\ \hline
	\textbf{用例描述} & \multicolumn{3}{p{12cm}}{该用例用来描述用户打开平台首页页面后,状态为未登录的用户通过输入账号和密码进行登录操作,系统将在登录过程中对用户账号及密码的正确性进行验证,并判断用户为个人用户还是企业用户,在登录后根据不同的账号类型在部分页面进行不同的页面显示。} \\ \hline
	\textbf{前置条件}& \multicolumn{3}{p{12cm}}{用户通过浏览器输入平台地址。} \\ \hline
	\textbf{基本事件流}& \multicolumn{3}{p{12cm}}{用户打开浏览器,在地址栏输入平台网址;\newline
		打开平台,展示平台首页;\newline
		用户账号为未登录状态,点击登录;\newline
		展示登录页面;\newline
		输入用户账号和密码进行验证。} \\ \hline
	\textbf{异常事件流}& \multicolumn{3}{p{12cm}}{1.网址URL 输入错误,打开平台首页失败;\newline
		2.用户账号或密码错误,登录失败。
	} \\ \hline
	\textbf{后置条件}& \multicolumn{3}{p{12cm}}{页面由登录页面回到首页,用户登录状态为已登录。} \\ \shline
\end{longtable}